\section{Gemma MLP}
\begin{figure}[h]
\centering
\begin{tikzpicture}[
    node distance=1.6cm,
    every node/.style={draw, rectangle, rounded corners, minimum height=0.9cm, minimum width=2.2cm, align=center},
    arrow/.style={->, thick}
]

% Nodes
\node (x) {$x$};
\node (gate) [right=of x] {$W_{\text{gate}} x$};
\node (up) [below=of gate] {$W_{\text{up}} x$};

\node (gelu) [right=of gate] {GELU};
\node (mul) [right=of up, yshift=0.8cm] {$\odot$};

\node (down) [right=of mul] {$W_{\text{down}}$};
\node (out) [right=of down] {Output};

% Arrows
\draw[arrow] (x) -- (gate);
\draw[arrow] (x) -- (up);

\draw[arrow] (gate) -- (gelu);
\draw[arrow] (gelu) -- (mul);

\draw[arrow] (up) -- (mul);

\draw[arrow] (mul) -- (down);
\draw[arrow] (down) -- (out);

\end{tikzpicture}
\caption{Gated MLP used in Gemma. The gate and up projections are fused in implementation but shown separately for clarity.}
\end{figure}

The Gemma MLP is a gated feed-forward network applied independently at each token position. For an input activation
\[
x \in \mathbb{R}^{d_{\text{model}}},
\]
the MLP computes
\[
\operatorname{MLP}(x)
=
W_{\text{down}}
\Bigl(
\operatorname{GELU}_{\tanh}(W_{\text{gate}} x)
\odot
(W_{\text{up}} x)
\Bigr),
\]
where
\[
W_{\text{gate}},\, W_{\text{up}} \in \mathbb{R}^{d_{\text{ff}} \times d_{\text{model}}},
\qquad
W_{\text{down}} \in \mathbb{R}^{d_{\text{model}} \times d_{\text{ff}}}.
\]

In implementation, the gate and up projections are fused into a single matrix multiplication:
\[
\begin{aligned}
\left[g(x),\,u(x)\right]
&=
\begin{bmatrix}
W_{\text{gate}} \\
W_{\text{up}}
\end{bmatrix}
x, \\
\operatorname{MLP}(x)
&=
W_{\text{down}}
\bigl(
\operatorname{GELU}_{\tanh}(g(x))
\odot u(x)
\bigr),
\end{aligned}
\]
reducing memory reads and improving throughput.

The activation function is the tanh-based approximation of GELU,
\[
\operatorname{GELU}_{\tanh}(z)
=
\frac{1}{2}z
\left(
1 + \tanh\!\left(
\sqrt{\frac{2}{\pi}}
\left(z + 0.044715\,z^3\right)
\right)
\right),
\]
which provides a smooth gating signal compared to other activation functions.
\begin{figure}[h]
    \centering
    \includegraphics[width=0.5\textwidth]{figures/gelu-tanh.png}
    \caption{Gelu and other activation functions}
    \label{fig:gelu-tanh}
\end{figure}

Compared to a standard two-layer feed-forward network,
\[
\operatorname{FFN}(x) = W_2\,\sigma(W_1 x),
\]
the gated structure introduces multiplicative interactions that allow the model to dynamically control feature flow. This increases expressivity without additional depth and is a key contributor to Gemma’s strong performance per parameter.

The MLP is used inside a pre-normalized residual block:
\[
h_{\ell+1}
=
h_\ell
+
\operatorname{MLP}\!\left(
\operatorname{RMSNorm}(h_\ell)
\right),
\]
ensuring stable optimization in deep Transformer stacks.
